%-----------------------------------------------------
\begin{alerttextbox}{Introductory Note}
This document is an adaption of the original datacamp.org cheat sheet.\\
\begin{itemize}
    \item https://www.datacamp.com/resources/cheat-sheets/pandas-cheat-sheet-for-data-science-in-python
    \item https://github.com/f616/Python-Pandas-Basics-Cheat-Sheet
\end{itemize}

\end{alerttextbox}


%-----------------------------------------------------
\begin{textbox}{Pandas}
The Pandas library is built on NumPy and provides easy-to-use data structures and data analysis tools for the Python programming language.
\end{textbox}

\begin{codebox}{python}{Use the following import convention:}
import pandas as pd
\end{codebox}

%--------------------------------------------------------------
\section{Pandas Data Structures}

\begin{myblock}{Series}
A \textbf{one-dimensional} labeled array capable of holding any data type\\\\

{index-->}\hspace{0.5cm}
\begin{tabular}{ |c|c| } 
\hline
 \cellcolor[HTML]{FFFFFF} a & 3\\
  \hline
 \cellcolor[HTML]{FFFFFF} b & -5\\
  \hline
 \cellcolor[HTML]{FFFFFF} c & 7\\
  \hline
 \cellcolor[HTML]{FFFFFF} d & 4\\
 \hline
\end{tabular}
\end{myblock}

\begin{codebox}{python}{}
s = pd.Series([3, -5, 7, 4], index=['a', 'b', 'c', 'd'])
\end{codebox}

\begin{myblock}{Dataframe}
A \textbf{two-dimensional} labeled data structure with columns of potentially different types\\\\

\begin{tabular}{ l|c||c|c|r| } 
\hhline{~----}
 columns--> &  & \cellcolor[HTML]{FFFFFF}  Country & \cellcolor[HTML]{FFFFFF}  Capital & \cellcolor[HTML]{FFFFFF}  Population\\
 \hhline{~====}
 index--> & \cellcolor[HTML]{FFFFFF} 0 & Belgium & Brussels & 11190846\\
 \hhline{~----}
  & \cellcolor[HTML]{FFFFFF} 1 & India & New Delhi & 1303171035\\
 \hhline{~----}
  & \cellcolor[HTML]{FFFFFF} 2 & Brazil & Brasília & 207847528\\
 \hhline{~----}
\end{tabular}
\end{myblock}

\begin{codebox}{python}{}
data = {'Country': ['Belgium', 'India', 'Brazil'], 'Capital': ['Brussels', 'New Delhi', 'Brasília'], 'Population': [11190846, 1303171035, 207847528]}
df = pd.DataFrame(data, columns=['Country', 'Capital', 'Population'])
\end{codebox}

%--------------------------------------------------------------
\section{Dropping}

\begin{codebox}{python}{}
s.drop(['a', 'c'])  #Drop values from rows (axis=0)
df.drop('Country', axis=1) #Drop values from columns(axis=1)
\end{codebox}


%--------------------------------------------------------------
\section{Asking For Help}

\begin{codebox}{python}{}
help(pd.Series.loc)
\end{codebox}


%--------------------------------------------------------------
\section{Sort \& Rank}

\begin{codebox}{python}{}
df.sort_index()  #Sort by labels along an axis
df.sort_values(by='Country')  #Sort by the values along an axis
df.rank()  #Assign ranks to entries
\end{codebox}


%--------------------------------------------------------------
\section{I/O}

\begin{codebox}{python}{Read and Write to CSV}
pd.read_csv('file.csv', header=None, nrows=5)
df.to_csv('myDataFrame.csv')
\end{codebox}

\begin{codebox}{python}{Read and Write to Excel}
xlsx = pd.ExcelFile('file.xlsx')
df = pd.read_excel(xlsx, 'Sheet1')  #Read from xlsx file Sheet1
df.to_excel('dir/myDataFrame.xlsx', sheet_name='Sheet1')  #Save to xlsx file
\end{codebox}

\begin{codebox}{python}{Read and Write to SQL Query or Database Table}
from sqlalchemy import create_engine
engine = create_engine('sqlite:///:memory:')
pd.read_sql("SELECT * FROM my_table;", engine)
pd.read_sql_table('my_table', engine)
pd.read_sql_query("SELECT * FROM my_table;", engine)

# read_sql() is a convenience wrapper around read_sql_table() and read_sql_query()

df.to_sql('myDf', engine)
\end{codebox}


%--------------------------------------------------------------
\section{Selection}

\begin{codebox}{python}{Getting}
s['b']  #Get one element
-5
df[1:]  #Get subset of a DataFrame
Country Capital Population
1 India New Delhi 1303171035
2 Brazil Brasília 207847528
\end{codebox}

\begin{myblock}{Selecting{,} Boolean Indexing \& Setting}
\textbf{By Position}
\begin{codebox}{python}{}
df.iloc[[0],[0]]  #Select single value by row & column
'Belgium'
df.iat[0,0]
'Belgium'
\end{codebox}

\textbf{By Label}
\begin{codebox}{python}{}
df.loc[[0], ['Country']]  #Select single value by row & column labels
'Belgium'
df.at[0, 'Country']
'Belgium'
\end{codebox}

\textbf{Boolean Indexing}
\begin{codebox}{python}{}
s[~(s > 1)]  #Series s where value is not >1
s[(s < -1) | (s > 2)]  #s where value is <-1 or >2
df[df['Population']>1200000000]  #Use filter to adjust DataFrame
\end{codebox}

\textbf{Setting}
\begin{codebox}{python}{}
s['a'] = 6  #Set index a of Series s to 6
\end{codebox}
\end{myblock}


%--------------------------------------------------------------
\section{Retrieving Series/DataFrame Information}

\begin{codebox}{python}{Basic Information}
df.shape  #(rows,columns)
df.index  #Describe index
df.columns  #Describe DataFrame columns
df.info()  #Info on DataFrame
df.count()  #Number of non-NA values
\end{codebox}

\begin{codebox}{python}{Summary}
df.sum()  #Sum of values
df.cumsum()  #Cummulative sum of values
df.min()  #Minimum values
df.max()  #Maximum values
df.idxmin()  #Minimum index value
df.idxmax()  #Maximum index value
df.mean()  #Mean of values
df.median()  #Median of values
df.describe()  #Summary statistics
\end{codebox}


%--------------------------------------------------------------
\section{Data Alignment}

\begin{myblock}{Internal Data Alignment}
\textbf{NA values are introduced in the indices that don’t overlap:}
\begin{codebox}{python}{}
s3 = pd.Series([7, -2, 3], index=['a', 'c', 'd'])
s + s3
a 10.0
b NaN
c 5.0
d 7.0
\end{codebox}
\end{myblock}

\begin{myblock}{Arithmetic Operations with Fill Methods}
\textbf{You can also do the internal data alignment yourself with the help of the fill methods:}
\begin{codebox}{python}{}
s.add(s3, fill_values=0)
a 10.0
b -5.0
c 5.0
d 7.0
s.sub(s3, fill_value=2)
s.div(s3, fill_value=4)
s.mul(s3, fill_value=3)
\end{codebox}
\end{myblock}
